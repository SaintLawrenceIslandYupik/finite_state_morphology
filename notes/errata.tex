\documentclass{article}

\setlength{\parindent}{0pt}
\usepackage{amsmath}
\usepackage{amssymb}
\usepackage{enumitem}
\usepackage{ulem}
\usepackage[utf8]{inputenc}

\begin{document}

\section{Possible Typos}

\begin{tabular}{p{1cm}p{10.5cm}}
p.24 & The surface form for "big river" is given as \textit{\textbf{kiiwhllagek}. Shouldn't the first undoubling rule apply?} \\

p.26 & The 1\textsuperscript{st} and 2\textsuperscript{nd} possessor possessed absolutive endings are missing the uvular drop colon that is included in Appendix I. \\

p.31 & The resulting form for \textit{tagi} + \textit{@lleqe} is given as \textit{\textbf{taqilleqe}} with a \textit{q}. \\

p.35 & The citation form for \textit{\textbf{apeghtughista}} is given as is without a subscript \textit{e}. In the EOC exercises however, \textit{apeghtughista} is conjugated as if its citation form is \textit{apeghtughiste}. \\

p.40 & The ablative-modalis form for \textit{\textbf{Sivuqaq}} is given multiple times as \textit{Sivuqaghmeng} in the EOC exercise for Chapter 6, but \textit{Sivuqaq}'s citation form has a weak \textit{gh} as seen on p.16 which should drop during formation of the ablative modalis.\\

p.42 & When affixing the postbase \textit{mii} to \textit{Sivuqaq}, Jacobson gives the form \textit{\textbf{Sivuqaghhmi}}.
%
Why is there a \textit{ghh}? \\

p.65 & 	Jacobson shows the 2nd person subject intransitive optative form for \textit{kenigh}, but writes earlier that \textit{kenigh} is transitive only.

If \textit{kenigh} is somehow available, Jacobson writes that the optative form of \textit{kenigh} is \textit{\textbf{kenii}} due to uvular dropping.
%
But uvular dropping doesn't take place unless the first vowel is a full vowel, correct? \\

p.92 & Jacobson says that \textit{amqeghte} - to bite is transitive, but gives an example where it is inflected with an intransitive ending. \\

p.119 & \textit{aqlaghaghtemmi} is the given form for \textit{aqlaghaghte} with a localis case ending, however, none of the localis endings contain \textit{mm}.
%
``For (possessor possessed) ablative-modalis, localis and terminalis the nek, ni, nun forms of the ending are used here rather than the mek, mi, mun forms even for singular possessed.''

\end{tabular}

\end{document}
