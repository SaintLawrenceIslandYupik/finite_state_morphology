\documentclass{article}

\usepackage[utf8]{inputenc}
\usepackage{amsmath}
\usepackage{color}
\usepackage{enumitem}
\usepackage{ulem}

\begin{document}

\section{Implementation Issues}

\begin{enumerate}

\item Is there a \textcolor{magenta}{lookahead feature} in \textit{foma}? This would particularly helpful with some postbases, such as @${\sim}_\text{f}$naqe, which when affixed to \textit{te}-verbs affect consonants that precede the \textit{te}, e.g. \textit{ingaghte} + @${\sim}_\text{f}$naqe $\rightarrow$ \textit{inga\uline{ghh}naqe}

\item Ideally, we would like to label the Optional Impersonal Agent verbs as such in their gloss.

\item According to Jacobson, the postbase \textit{(pete)fte} can attach as \textit{petefte} to consonant-ending bases or as \textit{pete} only.
%
While the transducer is analyzing both of these forms, it analyzes the former as if the postbase were affixing twice.
%
We will need to re-tweak these rules (See \textit{Juncture Consonant Insertion}).

\item The alternatives to the postbase \textit{@$\sim_{sf}$--(g)kaq/u)} haven't been implemented yet.

\item On p.71, it looks like those verbs associated with movement can take transitive endings in addition to intransitive.
%
Is this true?

\item What exactly are the postural root endings that appear in quantifier-qualifier constructions?

\item What is happening with the obsolete verb base \textit{ete}?

\item To form the plural \textsc{localis} and \textsc{terminalis} cases, final-\textit{e} does not drop from ``long'' bases.
%
The current final-\textit{e} dropping rule doesn't distinguish between long and short bases, because there is no need to except in this instance.
%
We'll need to figure out how best to handle this corner case, either with an additional final-\textit{e} dropping rule, an alternative path for the plural terminalis and localis, or in the \textit{Exceptions} file, etc.


\end{enumerate}


\section{Issues Pertaining to Overgeneration}

\item Bases \textit{negh} and \textit{megh} and their alternatives were all listed, and overgenerate when one of the bases is paired with a postbase that requires the alternative.

\item We are overgenerating the postbase \textit{ngugh*} which is used only with ``special'' \textit{te} bases.
%
Originally, we had implemented flag diacritics to handle this postbase, with appropriate flag diacritics on the relevant words, but there are some postbases that apply to all types of bases, including ``special'' \textit{te} bases.
%
By including a flag diacritic with these universal postbases, we would be excluding all other types of bases.
%
Essentially, we needed some other way to distinguish the special \textit{te} bases from the other bases, other than a flag diacritic, and we opted for marking them like so: \textit{te*}.

So the implementation of \textit{ngugh*} works, but it will apply to every base.
%
Including flag diacritics are also not ideal according to Aric, since they aren't intrinsically part of the transducer, are difficult to debug, and double the size of the transducer.
%
But maybe they should be included to prevent overgeneration in this instance.
%
Perhaps, we should include both the \textit{te*} marking and flag diacritics...

\item The postbase $\sim$sf--(g)ka/(g)ke, which uses different forms depending on whether or not the following person marker begins with \textit{ng}, is also overgenerating, since we have simply listed them both.
%
It is unclear how to predict the coming number, person marker, although flag diacritics might help.

This postbase can also only affix to possessed nouns, ones that include information about both the subject and object, and up until Chapter 11, is the only postbase of this sort.
%
This is not yet accounted for, and so this postbase is continuing to all number, person endings.
%
This issue also pertains to the postbase \textit{--lghii}, which should only affix to unpossessed nouns.

\item Once the above is resolved, a second pass should be made through the Participial endings which include all of the above postbases as its endings.
%
A possible solution may be to include \textit{--lghii}/\textit{ngugh*} in the MultipleForms slot of \textit{foma}.

\item Should a flag diacritic for \textsc{number} be included to account for conventional duals (maybe), and the postbase for ``N and company'', ``N and partner'', etc.?

\item \textbf{!!!Priority} We have a parsing issue with \textit{uyviinghet}.

\item For the conditional mood, verbs with a special \textit{te} ending have multiple permissible forms.
%
We need to check if this is true for other postbases that begin with \textit{k}, and perhaps restructure some rules to run in parallel.



\section{Vocabulary/Mood Ending Entry Issues}

\begin{itemize}
\renewcommand\labelitemi{$\cdot$}

\item Is \textit{\textbf{yughagh}} | to pray; to worship, intransitive or transitive?

\item Is the [2SgPoss][SgPosd] ending for \textsc{vialis} case, \textit{\textbf{--gpegun}} or \textit{\textbf{--gpekun}}?

\item On p.51, \textsc{[v][intrg][2du]} is given as \textit{\textbf{$\sim_\text{sf}$(e)stek}}, but in Appendix I, p.182, it is given as \textit{\textbf{$\sim_\text{sf}$(e)stekstek}}.

\item Which @ and (i) is it supposed to be in the transitive optative table?

\item We need to finish identifying and marking the special \textit{te} bases, as well as all the \textit{postural roots}, and the words that are used only in quantifier-qualifier constructions.

\item In Appendix I, Jacobson states that all \textit{p}-initial endings are for the participial.
%
Does this include both intransitive and transitive endings?

\item In Section 11.2.6, Jacobson says that \textit{(g)} is ``not usually'' used with bases that end in \textit{e}.
%
When is it used with those bases then?

\item The introduction of the concessive, conditional, contemporative endings in Chapter 14 don't match their corresponding table in Appendix I.
%
And when Jacobson writes \textit{like 3rd person object}, does he only mean the singular subject form?

\item What are the all the partial consonant assimilations that occur with enclitics?

\end{itemize}



\section{Big Picture Issues}

\begin{itemize}
\renewcommand\labelitemi{$\cdot$}

\item Standardize the glossing style and abbreviations.

\item Alphabetize and condense the nouns and verbs lists.

\end{itemize}

\end{document}

