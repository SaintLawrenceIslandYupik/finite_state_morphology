\documentclass{article}

\usepackage[utf8]{inputenc}
\usepackage{amsmath}
\usepackage{color}
\usepackage{enumitem}
\usepackage{ulem}

\begin{document}

\section{Implementation Issues}

\begin{itemize}
\renewcommand\labelitemi{$\cdot$}

\item In many explanations of postbases, Jacobson asserts that certain postbases can only continue on to certain endings (see Filters in \textit{ess.foma}).
%
Are these limitations long-distance in the sense that if an additional postbase appears between the ending and the limiting postbase, it does not affect the relationship that holds between the ending and the limiting postbase?

Note that implementing the transducer in this way \textit{\textbf{greatly}} increases the size of the transducer.
%
As a result, the relevant filters are all implemented in \textit{ess.foma}, but we have commented the problematic ones out so the transducer compiles faster for debugging purposes.

\item \textsc{chapter 16}: 16.2.8 which introduces the second terminalis ending is not yet implemented.

\end{itemize}

% % %

\subsection{Minor Implementation Issues}

\begin{itemize}
\renewcommand\labelitemi{$\cdot$}

\item Should a flag diacritic for \textsc{number} be included to account for conventional duals (although they can also be inflected for the singular), and the postbase for ``N and company'', ``N and partner'', etc.?

\item \textsc{chapter 8}: The postbase \textit{(pete)fte} can attach as \textit{petefte} or as \textit{pete} to consonant-ending bases.
%
The transducer successfully analyzes both forms, but also produces an underlying form where the postbase is affixed twice.
%
It assumes that \textit{pete} is affixed first as a result of the base ending in a consonant, followed by \textit{fte} since \textit{pete} ends in a vowel.

\item \textsc{chapter 15}: The half-transitive postbase takes on two forms, \textit{$\sim$i} and \textit{$\sim$:(u)te}.
%
Which one is used is dependent on the verb base itself, although there is no pattern to this, and ``a learner of the language must just learn which these verbs are; there is no way from the outside to predict''.

There are two other postbases of the form \textit{$\sim$:(u)te} that are applicable to all bases however.
%
They are distinguished from the half-transitive postbase only in the glossing of their underlying forms.
%
If we leave this distinction as is, then we should not apply a filter for the half-transitive postbase \textit{$\sim$:(u)te}, but we will end up overgenerating it.

\end{itemize}

% % % % % % % % % % % %

\section{Language-Related Issues}

\begin{itemize}
\renewcommand\labelitemi{$\cdot$}

\item \textsc{chapter 15}: Do all weakly special \textit{te}'s have \textit{t} become \textit{l} or \textit{s}?

\item \textsc{chapter 15}: Section 15.3.4  mentions an alternative use for the negative optative in place of the subordinative...?
%
Although these forms are already implemented in the negative optative section, they will presumably need to be re-glossed.

\item \textsc{chapter 17}: It's unclear what endings the postbase \textit{$\sim$-ke} continues on to.

\item \textsc{chapter 17}: Section 17.3.2 states that ``certain verb bases'' can take on unpossessed oblique case endings.
%
Aside from a few examples, Jacobson does not list all the verb bases that can do this, and gives very brief descriptions as to their resulting meaning.
%
This implementation and subsequent glossing needs to be revisited.

\item \textsc{Chapter 18}: Section 18.3.2 introduces ``augmentative'' postbases.
%
Are these ordinary postbases that belong in the appropriate Noun or Verb postbase section or should we list them and identify them separately as augmentative postbases, as they are now?

\item On p.71, it looks like those verbs associated with movement can take transitive endings in addition to intransitive.
%
Is this true?

\end{itemize}

% % %

\subsection{Vocabulary/Tense Ending Entry Issues}

\begin{itemize}

\item Is the [2SgPoss][SgPosd] ending for \textsc{vialis} case, \textit{\textbf{--gpegun}} or \textit{\textbf{--gpekun}}?

\item On p.51, \textsc{[v][intrg][2du]} is given as \textit{\textbf{$\sim_\text{sf}$(e)stek}}, but in Appendix I, p.182, it is given as \textit{\textbf{$\sim_\text{sf}$(e)stekstek}}.

\item What do @ and (i) do in the transitive optative endings?

\item The introduction of the concessive, conditional, contemporative endings in Chapter 14 don't match their corresponding table in Appendix I.

\item What are the all the partial consonant assimilations that occur with enclitics?

\item What are all the words that belong to the postinflectional class, e.g. \textit{mekestaaghhaq}, discussed in 18.3.4?

\item What are the endings to the volitive of fear?

\end{itemize}

% % % % % % % % % % % %

\section{TODO}

\begin{itemize}
\renewcommand\labelitemi{$\cdot$}

\item \textsc{chapter 12}: What is happening with the obsolete verb base \textit{ete}?

\item Review the implementation of question words and their continuation classes.

\item We have a tokenization issue with \textit{uyviinghet}.

\item What is happening with \textit{tuququghii}?

\end{itemize}

% % % % % % % % % % % %

\section{Big Picture Issues}

\begin{itemize}
\renewcommand\labelitemi{$\cdot$}

\item The demonstrative system is currently hard-coded into the \textit{lexc} file, while the numeral system has not yet been implemented at all.
%
More work/elicitation needs to be done regarding these to systems, to better understand how they inflect and what their proper continuation classes should be.

\item We need to finish identifying and marking the special \textit{te} bases, as well as all the postural roots, emotional roots, and the words that are used only in quantifier-qualifier constructions.

\item Standardize the glossing style and abbreviations.
%
(Would removing the postbases' semantic glosses minimize the transducer?)

\item Add our glosses to the \textit{design.tex} file under \textbf{Glossary of Morphophonogical Symbols}, and the \textit{yupik\_plans} repository.

\item Alphabetize and condense the nouns and verbs lists.

\item Enter the combining Unicode characters to replace \textit{$\sim$}, \textit{$\sim$f}, and \textit{$\sim$sf}.

\item Check ``aesthetics'' of the \textit{foma} and \textit{lexc} files, e.g. tables are properly formatted, environments are aligned, etc.

\end{itemize}

\end{document}
