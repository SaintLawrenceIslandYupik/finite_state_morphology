\documentclass{article}

\usepackage[utf8]{inputenc}
\usepackage{amsmath}
\usepackage{color}
\usepackage{enumitem}
\usepackage{ulem}

\begin{document}

% % % % % %

\section{Implementation Issues / Questions for Mans}

\begin{itemize}
\renewcommand\labelitemi{$\cdot$}

\item Bases \textit{negh} and \textit{megh} and their alternatives, \textit{neghe}, \textit{meghe} are all listed, and overgenerate when one of the bases is paired with a postbase that requires the alternative.
%
That being said, is it possible to have ``partial'' exceptions in the \textit{Exceptions} file?

\item We have a tokenization issue with \textit{uyviinghet}.

\item \textbf{p.72: 11.2.6}: Is there a ``lookahead'' feature in \textit{foma}?
%
This would prevent overgeneration of the postbase \textbf{@$\sim_\text{sf}$--(g)ke/(g)ka}, where the latter is used only if the possessive ending starts with \textit{ng}.

\item The NounPostbase and VerbPostbase classes loop on themselves, which is acceptable and necessary for the most part, except when they loop in such a way that postbases are repeated.
%
Is there some way to prevent this, as it gets excessive... (Apply up to \textit{tuququghii}).

\end{itemize}

\subsection{Issues Pertaining To Overgeneration}

\begin{itemize}
\renewcommand\labelitemi{$\cdot$}

\item \textsc{chapter 11}: We are overgenerating the postbase \textbf{ngugh*} which is used only with ``special'' \textit{te} verb bases.
%
Originally, flag diacritics were implemented to account for special \textit{te} bases, but nearly all other postbases apply to all types of bases, including special \textit{te}.
%
Would flag diacritics be advisable in this case, since they have a tendency to double the size of the transducer?

\item \textsc{chapter 11}: The above postbase \textbf{@$\sim_\text{sf}$--(g)ke/(g)ka} should only continue on to the possessed nominal endings.
%
Likewise, the postbase \textbf{--lghii} should only continue on to the unpossessed nominal endings.

\item \textsc{chapter 15}: The half-transitive postbase takes on two forms, \textit{$\sim$i} and \textit{$\sim$:(u)te}.
%
Which one is used is dependent on the verb base itself, although there is no pattern to this, and ``a learner of the language must just learn which these verbs are; there is no way from the outside to predict''.

\item \textsc{chapter 17}: The postbase \textit{$\sim$:(u)te/ta} has two forms, where the latter is used only with 3rd person possessor possessed endings.

\item \textsc{chapter 17}: Section 17.3.2 states that ``certain verb bases'' can take on unpossessed oblique case endings.
%
Jacobson does not specify which verb bases may do so, so one not so nice option may be to simply list the unpossessed oblique case endings in the VerbTenseInfl class, and simply permit every verb to potentially affix an oblique case.

\item \textsc{chapter 18}: 18.3.2 introduces how certain augmentive postbases can be used directly with verbs, and continue on to \textit{only} unpossessed absolutive endings.

\item \textsc{chapter 18}: 18.3.6 introduces how the plural localis can be used with the postbase \textit{--lghii} to mean \textit{when one is V-ing}.
%
This is a very specific path through the transducer...

\end{itemize}

% % % % % % % % % % % %

\section{Vocabulary/Mood Ending Entry Issues}

\begin{itemize}
\renewcommand\labelitemi{$\cdot$}

\item Is \textit{\textbf{yughagh}} | to pray; to worship, intransitive or transitive?

\item Is the [2SgPoss][SgPosd] ending for \textsc{vialis} case, \textit{\textbf{--gpegun}} or \textit{\textbf{--gpekun}}?

\item On p.51, \textsc{[v][intrg][2du]} is given as \textit{\textbf{$\sim_\text{sf}$(e)stek}}, but in Appendix I, p.182, it is given as \textit{\textbf{$\sim_\text{sf}$(e)stekstek}}.

\item Which @ and (i) is it supposed to be in the transitive optative table?

\item In Appendix I, Jacobson states that all \textit{p}-initial endings are for the participial.
%
Does this include both intransitive and transitive endings?

\item The introduction of the concessive, conditional, contemporative endings in Chapter 14 don't match their corresponding table in Appendix I.
%
And when Jacobson writes \textit{like 3rd person object}, does he only mean the singular subject form?

\item What are the all the partial consonant assimilations that occur with enclitics?

\item On p.71, it looks like those verbs associated with movement can take transitive endings in addition to intransitive.
%
Is this true?

\item What exactly are the postural root endings that appear in quantifier-qualifier constructions?

\item What are all the words that belong to the postinflectional class, e.g. \textit{mekestaaghhaq}, discussed in 18.3.4?

\item What are the endings to the volitive of fear?

\end{itemize}

% % % % % %

\section{Corner Cases That Need Implementing/Debugging}

\begin{itemize}
\renewcommand\labelitemi{$\cdot$}

\item \textsc{chapter 5}: To form the plural \textsc{localis} and \textsc{terminalis} cases, final-\textit{e} does not drop from ``long'' bases.
%
The current final-\textit{e} dropping rule doesn't distinguish between long and short bases, because there is no need to except in this instance.
%
We'll need to figure out how best to handle this corner case, either with an additional final-\textit{e} dropping rule, an alternative path for the plural terminalis and localis, or in the \textit{Exceptions} file, etc.

\item \textsc{chapter 8}: According to Jacobson, the postbase \textit{(pete)fte} can attach as \textit{petefte} to consonant-ending bases or as \textit{pete} only.
%
While the transducer is analyzing both of these forms, it analyzes the former as if the postbase were affixing twice.

\item \textsc{chapter 11}: Review the Participial endings, since we might be able to condense and simplify them.
%
A possible solution may be to include \textit{--lghii}/\textit{ngugh*} in the MultipleForms slot of \textit{foma}.

\item \textsc{chapter 12}: What is happening with the obsolete verb base \textit{ete}?

\item The alternatives to the postbase \textit{@$\sim_{sf}$--(g)kaq/kau)} haven't been implemented yet.

\item Should a flag diacritic for \textsc{number} be included to account for conventional duals (maybe), and the postbase for ``N and company'', ``N and partner'', etc.?

\item \textsc{chapter 14}: For the conditional mood, verbs with a special \textit{te} ending have multiple permissible forms.

\item \textsc{chapter 15}: Do all weakly special \textit{te}'s have \textit{t} become \textit{l} or \textit{s}?

\item \textsc{chapter 15}: Section 15.3.4 is already implemented in the optative section, but the glossing is incorrect as a result.

\item \textsc{chapter 15}: There is something wrong with the implementation of strongly special \textit{te}'s, in that once either the \textit{@--nghite} or \textit{~--(ng)ite} postbase is followed by the conditional tense, the \textit{t} does not turn into \textit{l}.
%
Affixing the participial tense however does not cause this problem.

\item \textsc{chapter 16}: 16.2.8 which introduces the second terminalis ending is not yet implemented.

\item \textsc{chapter 17}: It's unclear what endings the postbase \textit{$\sim$-ke} continues on to.

\end{itemize}

% % % % % %

\section{Big Picture Issues}

\begin{itemize}
\renewcommand\labelitemi{$\cdot$}

\item The demonstrative system is currently hard-coded into the \textit{lexc} file, while the numeral system has not yet been implemented at all.
%
More work/elicitation needs to be done regarding these to systems, to better understand how they inflect and what their proper continuation classes should be.

\item We need to finish identifying and marking the special \textit{te} bases, as well as all the postural roots, emotional roots, and the words that are used only in quantifier-qualifier constructions.

\item Standardize the glossing style and abbreviations.

\item Add our glosses to the \textit{design.tex} file under \textbf{Glossary of Morphophonogical Symbols}, and the \textit{yupik\_plans} repository.

\item Alphabetize and condense the nouns and verbs lists.

\item Check ``aesthetics'' of the \textit{foma} and \textit{lexc} files, e.g. tables are properly formatted, environments are aligned, etc.

\end{itemize}

\end{document}
