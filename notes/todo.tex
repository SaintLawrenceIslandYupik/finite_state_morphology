\documentclass{article}

\usepackage[utf8]{inputenc}
\usepackage{amsmath}
\usepackage{color}
\usepackage{enumitem}
\usepackage{ulem}

\begin{document}

\section{TODO}

\begin{enumerate}
\item It appears that the possessive absolutive forms can qualify as derivational morphemes as well as inflectional morphemes as they are classified now in the \textit{lexc} file (See \textit{\textbf{nuyaqatakestaaghhaaguq}} in the EOC exercises for Chapter 4).
%
We need to implement these forms as derivational morphemes, but also optimize the network to prevent circular, repetitive paths.
%
These result since the inflectional morphemes class is a continuation class of the derivational morpheme class.
%
If a morpheme classifies as both, one could potentially affix it twice as we move through the network.

\textcolor{magenta}{According to misc grammar notes in Chapter 18, it appears there are a limited number of postbases that can attach to an inflected noun.
%
Perhaps these unique postbases can be listed in a different continuation class from the other postbases, possibly with enclitics.}

\item Is there a \textcolor{magenta}{lookahead feature} in \textit{foma}? This would particularly helpful with some postbases, such as @${\sim}_\text{f}$naqe, which when affixed to \textit{te}-verbs affect consonants that precede the \textit{te}, e.g. \textit{ingaghte} + @${\sim}_\text{f}$naqe $\rightarrow$ \textit{inga\uline{ghh}naqe}

\item The absolutive case does not bear an \textit{unpossessed} marker like the other cases. \textcolor{magenta}{Should the [\textsc{unpd}] marker be added to the absolutive or simply removed entirely?} \ldots Add it.

\item Ideally, we would like to label the Optional Impersonal Agent verbs as such in their gloss.

\end{enumerate}


\section{Minor Issues}

\begin{itemize}
\renewcommand\labelitemi{$\cdot$}

\item Is \textit{\textbf{yughagh}} | to pray; to worship, intransitive or transitive?

\item To form the plural \textsc{localis} and \textsc{terminalis} cases, final-\textit{e} does not drop from ``long'' bases.
%
The current final-\textit{e} dropping rule doesn't distinguish between long and short bases, because there is no need to except in this instance.
%
We'll need to figure out how best to handle this corner case, either with an additional final-\textit{e} dropping rule, an alternative path for the plural terminalis and localis, or in the \textit{Exceptions} file, etc.

\item Is the [2SgPoss][SgPosd] ending for \textsc{vialis} case, \textit{\textbf{--gpegun}} or \textit{\textbf{--gpekun}}?

\item On p.51, \textsc{[v][intrg][2du]} is given as \textit{\textbf{$\sim_\text{sf}$(e)stek}}, but in Appendix I, p.182, it is given as \textit{\textbf{$\sim_\text{sf}$(e)stekstek}}.

\item The issue with \textit{\textbf{tipele}} regarding which \textit{e} is dropped when confronted with both semi-final and final-\textit{e} dropping.

\item According to Jacobson, the postbase \textit{(pete)fte} can attach as \textit{petefte} to consonant-ending bases or as \textit{pete} only.
%
While the transducer is analyzing both of these forms, it analyzes the former as if the postbase were affixing twice.
%
We will need to re-tweak these rules (See \textit{Juncture Consonant Insertion}).

\item The alternatives to the postbase \textit{@$\sim_{sf}$--(g)kaq/u)} haven't been implemented yet, since the postbase itself needs some tweaking (See order of operations concerning semi-final \textit{e} dropping, final \textit{e} dropping, final consonant drop assimilation).

\end{itemize}


\end{document}

