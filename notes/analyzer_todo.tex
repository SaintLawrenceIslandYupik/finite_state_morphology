\documentclass{article}

\usepackage[a4paper]{geometry}


\begin{document}

%%%%%%%%%%%%%%%%%%%%%%
%                    %
%   Logical Errors   %
%                    %
%%%%%%%%%%%%%%%%%%%%%%

\section{Logical Errors}

\begin{itemize}
\renewcommand\labelitemi{$\cdot$}

\item There seem to be recurring issues when postbases are affixed to the underlying form of \textbf{kiikw} (See \textit{kiiw+te}).

\item \textbf{aate@$\sim$:(u)sigh-} generates \textit{aaasigh-}

\item \textbf{qeneghte-@₁~:(u)tke-} generates \textit{qenghutke-}, which is correct, but will not parse in the other direction (from surface to underlying).
%
The dictionary also permits \textit{qeneghtutke-} where semi-final \textit{e} is not dropped.


\end{itemize}


%%%%%%%%%%%%%%%%%%%%%%%%%%%
%                         %
%   Not Yet Implemented   %
%                         %
%%%%%%%%%%%%%%%%%%%%%%%%%%%

\section{Not Yet Implemented}

\subsection{Postbases}

\begin{itemize}
\renewcommand\labelitemi{$\cdot$}

\item \textbf{+tmiighte} \textit{to go in N direction} . Used with demonstrative adverb bases and positionals, but the latter appear to be fully lexicalized  

\item \textbf{+tmun} \textit{N-ward, toward N} .  Used with demonstrative adverb bases and positionals, but the latter appear to be fully lexicalized  

\item \textbf{@$\sim$:(u)ta} \textit{condition with respect to V-ing} . Non-productive unless it carries the meaning listed here

\end{itemize}


%%%%%%%%%%%%%%%%%%%%%%%%%%%%%
%                           %
%   Overgeneration Issues   %
%                           %
%%%%%%%%%%%%%%%%%%%%%%%%%%%%%

\section{Overgeneration Issues}

%%%%%%%%%%%%%%%%%%%
%                 %
%   T-Postbases   %
%                 %
%%%%%%%%%%%%%%%%%%%
\subsection{\textbf{T}-Postbases}

\begin{itemize}
\renewcommand\labelitemi{$\cdot$}

\item \textbf{+ta} \textit{N area of possessor} . Used with demonstrative adverb bases and takes possessed endings

\item \textbf{+tagh} \textit{for N to continuously occur} . Used with words denoting sounds

\item \textbf{+tala} \textit{the extent of V-ness} . Used with descriptive bases

\item \textbf{+tutka} \textit{to be V or have N to an indicated degree} . Used with descriptive bases

\end{itemize}

%%%%%%%%%%%%%%%%%%%
%                 %
%   U-Postbases   %
%                 %
%%%%%%%%%%%%%%%%%%%
\subsection{\textbf{U}-Postbases}

\begin{itemize}
\renewcommand\labelitemi{$\cdot$}

\item \textbf{$\sim_\text{f}$ughtagh} \textit{to V all things available; to completely V} . Used with bases that end in \textit{-e} 

\end{itemize}

%%%%%%%%%%%%%%%%%%%
%                 %
%   V-Postbases   %
%                 %
%%%%%%%%%%%%%%%%%%%
\subsection{\textbf{V}-Postbases}

\begin{itemize}
\renewcommand\labelitemi{$\cdot$}

\item \textbf{?vagh} \textit{to go N-wards; to put N-wards; to go or put toward N} . Used with positional bases and some demonstrative adverb bases\footnote{When used with \textbf{qule} \textit{area above}, it drops and hopes final-\textit{e}}

\end{itemize}



%%%%%%%%%%%%%%%%%%%%%%%%%%%%%%%%%%%%%%%%%%%%%%%%%%
%                                                %
%   Questions for Native  Speakers / Fieldwork   % 
%                                                %
%%%%%%%%%%%%%%%%%%%%%%%%%%%%%%%%%%%%%%%%%%%%%%%%%%

\section{Questions for Native Speakers / Fieldwork}

%%%%%%%%%%%%%
%           %
%   Nouns   %
%           %
%%%%%%%%%%%%%

\subsection{Questions Concerning Nouns}

\begin{itemize}
\renewcommand\labelitemi{$\cdot$}

\item Which nouns/nominal postbases that end in \textit{-a} in the citation form end in \textit{-e} in the underlying form? 

\item Which nouns/nominal postbases ends in strong \textit{-gh} / weak \textit{-gh}? 

\end{itemize}

%%%%%%%%%%%%%%%%%
%               %
%   Postbases   %
%               %
%%%%%%%%%%%%%%%%%
\subsection{Questions Concerning Postbases}

\begin{itemize}
\renewcommand\labelitemi{$\cdot$}

\item The dictionary documents several postbases as being allomorphs of one another, e.g. \textit{mlaagh}, \textit{mraagh}, \textit{vlaagh}, etc.
%
Are there no conditions on when each form surfaces?

\end{itemize}


%%%%%%%%%%%%%%%%%%%%%
%                   %
%   Design Choices  % 
%                   %
%%%%%%%%%%%%%%%%%%%%%

\section{Notes on Design Choices}

\begin{itemize}
\renewcommand\labelitemi{$\cdot$}

\item Postbases marked \textit{marginally-productive} were implemented, while those that were marked \textit{probably non-productive} were implemented if the examples given did not seem to be lexicalized.
%
Those that were marked \textit{non-producitve} were removed.

\item Some postbase-attachment examples introduced bases that were not included in the original lexicon of bases (See \textit{+tala}).

\end{itemize}

\end{document}
