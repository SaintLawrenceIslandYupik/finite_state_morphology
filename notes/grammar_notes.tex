\documentclass{article}

\setlength{\parindent}{0pt}
\usepackage{amsmath}
\usepackage{amssymb}
\usepackage{enumitem}
\usepackage{ulem}
\usepackage[utf8]{inputenc}

\begin{document}

\section{Possible Typos}

\begin{tabular}{p{1cm}p{10.5cm}}
p.24 & The surface form for "big river" is given as \textit{\textbf{kiiwhllagek}. Shouldn't the first undoubling rule apply?} \\

p.26 & The 1\textsuperscript{st} and 2\textsuperscript{nd} possessor possessed absolutive endings are missing the uvular drop colon that is included in Appendix I. \\

p.31 & The resulting form for \textit{tagi} + \textit{@lleqe} is given as \textit{\textbf{taqilleqe}} with a \textit{q}. \\

p.35 & The citation form for \textit{\textbf{apeghtughista}} is given as is without a subscript \textit{e}. In the EOC exercises however, \textit{apeghtughista} is conjugated as if its citation form is \textit{apeghtughiste}. \\

p.40 & The ablative-modalis form for \textit{\textbf{Sivuqaq}} is given multiple times as \textit{Sivuqaghmeng} in the EOC exercise for Chapter 6, but \textit{Sivuqaq}'s citation form has a weak \textit{gh} as seen on p.16 which should drop during formation of the ablative modalis.\\

\end{tabular}

%%%%%%%%%%%%%%%%%%%%%%%%%%%%%%%%%%%%%%%%%%
%               CHAPTER 2                %
%   Bases, Unpossessed Plural and Dual   %
%%%%%%%%%%%%%%%%%%%%%%%%%%%%%%%%%%%%%%%%%%
\section*{CHAPTER 2}
For base to citation form conversions, see Section 2.1 in Jacobson (2001)

\bigskip

\textsc{[n][abs][pl]}: \textbf{${\sim}_\text{sf}\text{-}_\text{w}$:(e)t}

\noindent \textsc{[n][abs][du]}: \textbf{${\sim}_\text{sf}\text{-}_\text{w}$:(e)k}

\bigskip

\textit{Note}: Some nouns only take the dual form $\rightarrow$ Implement \textit{flag diacritics}

%%%%%%%%%%%%%%%%%%%%%%%%%%%%%%%%%%%%%%%%%%%%%%%%%%%%%%%%%%%%%
%                         CHAPTER 3                         %
%   Intransitive Indicative, Unpossessed Ablative-Modalis   %
%%%%%%%%%%%%%%%%%%%%%%%%%%%%%%%%%%%%%%%%%%%%%%%%%%%%%%%%%%%%%
\section*{CHAPTER 3}

\noindent \textsc{[v][intr\_ind]}: \textbf{${\sim}_\text{f}$(g/t)u-}

\bigskip

\textit{Person/Number Markings for Intransitive Indicative}

\begin{tabular}{ l l }
\textbf{-q} & he, she, it \\
\textbf{-t} & they$_\text{\textsc{pl}}$ \\
\textbf{-k} & they$_\text{2}$ \\
\textbf{-nga} & I \\
\textbf{-kut} & we$_\text{\textsc{pl}}$ \\
\textbf{-kung} & we$_\text{2}$ \\
\textbf{-ten} & you$_\text{1}$ \\
\textbf{-si} & you$_\text{\textsc{pl}}$ \\
\textbf{-tek} & you$_\text{2}$
\end{tabular}

\bigskip

\textit{Note}: Some verb bases are presented with angled brackets around the final `\textit{e}', e.g. \textbf{negh$\langle$e$\rangle$-} and \textbf{megh$\langle$e$\rangle$-}.

This indicates that with some endings, these verbs behave as \textbf{negh-} and \textbf{megh-}, and with others as \textbf{neghe-} and \textbf{meghe-} $\rightarrow$ Implement in \textit{Exceptions.lexc}

\bigskip

\textsc{[n][unpd][abl\_mod]}: \textbf{$\sim_\text{f}$-$_\text{w}$meng}

%%%%%%%%%%%%%%%%%%%%%%%%%%%%%%%%%%%%%%%%%%%%%%%%%%%%%%%%%%%%%%
%                         CHAPTER 4                          %
%   1st and 2nd Person Possessor Possessed Absolutive Case   %
%%%%%%%%%%%%%%%%%%%%%%%%%%%%%%%%%%%%%%%%%%%%%%%%%%%%%%%%%%%%%%
\section*{CHAPTER 4}

\textit{Postbases}
\begin{description}[nolistsep, noitemsep]
\item \textbf{--ghllak}: big N, lots of N
\item \textbf{--ghrugllak}: big or huge N
\item \textbf{$\sim$:(ng)u}: to be N
\item \textbf{--lek}: one having N
\end{description}

\bigskip

\textit{Singular Possessor} \\
\begin{tabular}{ l l }
\textbf{$\sim$-ka} & my N \\
\textbf{${\sim}_\text{sf}\text{-}_\text{w}$:(e)nka} & my N$_\text{\textsc{pl}}$ \\
\textbf{${\sim}_\text{sf}\text{-}_\text{w}$:(e)gka} & my N$_\text{\textsc{2}}$ \\
\textbf{${\sim}_\text{sf}\text{-}_\text{w}$:(e)n} & your$_1$ N \\
\textbf{+ten} & your$_1$ N$_\text{\textsc{pl}}$ \\
\textbf{${\sim}_\text{sf}\text{-}_\text{w}$:(e)gken} & your$_1$ N$_\text{\textsc{2}}$ \\
\end{tabular}

\bigskip

\textit{Note}: The first ending, \textit{-ka}, drops final and semi-final `\textit{e}', and \textit{hops them if possible} $\rightarrow$ Implemented as `$\sim$h'

If the dropped final consonant is \underline{uvular}, the possessor ending becomes \textit{-qa}.

If the dropped final consonant is \underline{rounded}, the ending becomes \textit{-kwa} or \textit{-qwa}.

\bigskip

\textit{Plural Possessor} \\
\begin{tabular}{ l l }
\textbf{+put/+vut} & our N \\
\textbf{--put} & our N$_\text{\textsc{pl}}$ \\
\textbf{+si/+zi} & your$_\text{\textsc{pl}}$ N \\
\textbf{--si} & your$_\text{\textsc{pl}}$ N$_\text{\textsc{pl}}$ \\
\end{tabular}

\textit{Note}: The forms \textit{$+$put} and \textit{$+$si} are used with bases that end in a consonant, while \textit{$+$vut} and \textit{$+$zi} are used with bases that end in a vowel.

%%%%%%%%%%%%%%%%%%%%%%%%%%%%%%%%%%%%%%%%%%%%%%%%%%%%%%%%%%%%%%%%%%
%                           CHAPTER 5                            %
%   Unpossessed Localis, Terminalis, Vialis, and Equalis Cases   %
%%%%%%%%%%%%%%%%%%%%%%%%%%%%%%%%%%%%%%%%%%%%%%%%%%%%%%%%%%%%%%%%%%
\section*{CHAPTER 5}

\textit{Postbases}
\begin{description}[nolistsep, noitemsep]
\item \textbf{--(g)aqe}: to be V-ing; to regularly or repeatedly V
\item \textbf{@lleqe}: to V in the future; will V
\item \textbf{@$\sim_\text{f}$naqe}: to be going to V, to be about to V
\item \textbf{@$\sim_\text{f}$yug}: to want to V
\end{description}

\bigskip

\textsc{[unpd][loc][sg]}: \textbf{$\sim_\text{f}$-$_\text{w}$mi}

\textsc{[unpd][ter][sg]}: \textbf{$\sim_\text{f}$-$_\text{w}$mun}

\textit{Note}: The \textsc{plurals} of these cases have \textit{n} in place of \textit{m}, but final \textit{e} is dropped only from ???longer bases.

\bigskip

\textsc{[unpd][via][sg]}: \textbf{$\sim_\text{f}$-$_\text{w}$kun}

\textsc{[unpd][via][pl]}: \textbf{$\sim_\text{sf}$-$_\text{w}$:(e)tgun}

\textsc{[unpd][equ][sg]}: \textbf{$\sim_\text{f}$-$_\text{w}$tun}

\textsc{[unpd][loc][pl]}: \textbf{$\sim_\text{sf}$-$_\text{w}$:(e)stun}

%%%%%%%%%%%%%%%%%%%%%%%%%%%%%%%%%%%%%%%%%%%%%%%%%%%%%%%%%%%%%%%%%%%%%%%%%%
%                                 CHAPTER 6                              %
%   3rd Possessor Possessed Absolutive Case, Unpossessed Relative Case   %
%%%%%%%%%%%%%%%%%%%%%%%%%%%%%%%%%%%%%%%%%%%%%%%%%%%%%%%%%%%%%%%%%%%%%%%%%%
\section{CHAPTER 6}

\textit{Third Person Possessor} \\
\begin{tabular}{ l l }
\textbf{$\sim$:(ng)a} & his or her or its N \\
\textbf{$\sim$:(ng)at} & their N \\
\textbf{$\sim$:(ng)i} & his N$_\text{\textsc{pl}}$ \\
\textbf{$\sim$:(ng)it} & their N$_\text{\textsc{pl}}$ \\
\textbf{${\sim}_\text{f}\text{-}_\text{w}$kek} & his N$_\text{\textsc{2}}$ \\
\end{tabular}

\bigskip

\textsc{[unpd][rel][sg]}: \textbf{$\sim_\text{sf}$-$_\text{w}$(e)m}

\textsc{[unpd][loc][pl]}: \textbf{$\sim_\text{sf}$-$_\text{w}$(e)t}

%%%%%%%%%%%%%%%%%%%%%%%%%%%%%%%%%%%%%%%%%%%%%%%%%%%%%%%%%%%%%%%%%%%%%%%%%%%%%%%%%%%%%%%%
%                                      CHAPTER 7                                       %
%   Transitive Indicative, Possessed Relative, Possessed Ablative-Modalis, LOC, etc.   %
%%%%%%%%%%%%%%%%%%%%%%%%%%%%%%%%%%%%%%%%%%%%%%%%%%%%%%%%%%%%%%%%%%%%%%%%%%%%%%%%%%%%%%%%
\section{CHAPTER 7}

\textit{Postbases}
\begin{description}[nolistsep, noitemsep]
\item \textbf{+e}: used to make English words into Yupik words
\item \textbf{--ghhalek}: how V! (exclamation)
\item \textbf{+mii}: resident of N
\item \textbf{@--nghite}: to not V
\item \textbf{--ngllagh}: to make N
\item \textbf{--squq, --squghhaq*}: little N, small N
\end{description}

\bigskip

\textit{Possessed Relative Case} \\
\begin{tabular}{ l l }
\textbf{$\sim_\text{f}\text{-}_\text{w}$ma} & my N or N$_\text{\textsc{pl}}$ \\
\textbf{--mta} &  our N or N$_\text{\textsc{pl}}$\\
\textbf{--gpek} & your$_1$ N or N$_\text{\textsc{pl}}$ \\
\textbf{--gpesi} & your$_text{pl}$ N or N$_\text{\textsc{pl}}$ \\
\textbf{$\sim$:(ng)an} & his N or N$_\text{\textsc{pl}}$ \\
\textbf{$\sim$:(ng)ita} & their N or N$_\text{\textsc{pl}}$ \\
\end{tabular}

\bigskip

\textsc{[v][trns\_ind]}: \textbf{$\sim$(g)a}

\bigskip

\textit{Person/Number Markings for Transitive Indicative}

\begin{tabular}{ l l }
\textbf{-a} & he, she, or it to him, her, or it\\
\textbf{-i} & he to them \\
\textbf{-at} & they to him \\
\textbf{-it} & they to them \\
\textbf{-qa} & I to him \\
\textbf{-nka} & I to them \\
\textbf{-n} & you to him \\
\textbf{-ten} & you to them \\
\textbf{-anga} & he to me \\
\textbf{-aten} & he to you \\
\textbf{-mken} & I to you \\
\textbf{-ghpenga} & you to me
\end{tabular}

%%%%%%%%%%%%%%%%%%%%%%%%%%%%%%%%%%%%%%%%%%%%%%%%%%%%%%%%%%%%%%%%%%
%                             CHAPTER 8                          %
%   2nd Person Subject Interrogative, 4th Person v. 3rd Person   %
%%%%%%%%%%%%%%%%%%%%%%%%%%%%%%%%%%%%%%%%%%%%%%%%%%%%%%%%%%%%%%%%%%%
\section{CHAPTER 8}

\textit{Postbases}
\begin{description}[nolistsep, noitemsep]
\item \textbf{+(pete)fte}: to evidently V or have V-ed
\item \textbf{--liigh}: to cook; to prepare N
\item \textbf{+niigh}: to hunt for N; to search for N; to ask for N; to work with N
\item \textbf{+te}: to catch N; to go to N
\item \textbf{+tugh}: to eat N; to use N
\end{description}

\bigskip

\textit{Second Person Subject Interrogatives for Intransitives}

\begin{tabular}{ l l }
\textbf{$\sim_\text{f}$(t)zin} & you$_1$\\
\textbf{$\sim_\text{sf}$(e)tsi} & you$_\text{\textsc{pl}}$ \\
\textbf{$\sim_\text{sf}$(e)stek} & you$_2$ \\
\textbf{$\sim_\text{sf}$(e)sta} & we$_\text{\textsc{pl}}$ \\
\textbf{$\sim_\text{sf}$(e)stung} & we$_\text{\textsc{2}}$
\end{tabular}

\bigskip

\textit{Second Person Subject Interrogatives for Transitives}

\begin{tabular}{ l l }
\textbf{$\sim_\text{f}$(t)zigu} & you$_1$ to him, her, it\\
\textbf{$\sim_\text{f}$(t)ziki} & you$_1$ to them$_\text{\textsc{pl}}$ \\
\textbf{$\sim_\text{f}$(t)zikek} & you$_1$ to them$_\text{\textsc{2}}$ \\
\textbf{$\sim_\text{f}$(t)zinga} & you$_1$ to me \\
\textbf{$\sim_\text{f}$(t)zikut} & you$_1$ to us$_\text{\textsc{pl}}$
\end{tabular}

\bigskip

\textsc{[n][abs][4sgposs][sg/plposd]}: \textbf{--ni}

\textsc{[n][rel][4sgposs][sg/plposd]}: \textbf{$\sim_\text{f}\text{-}_\text{w}$mi}

\textsc{[n][abl\_mod][4sgposs][sg/plposd]}: \textbf{$\sim_\text{f}\text{-}_\text{w}$mineng}

%%%%%%%%%%%%%%%%%%%%%%%%%%%%%%%%%%%%%%%%%%%%%%%%%%%%%%%%%%%%%%%%%%%%%%%%%%%
%                                    CHAPTER 9                            %
%  Compound-Verbal Postbases, Optional Impersonal Agent Verbs, 3SBJ Intrg %
%%%%%%%%%%%%%%%%%%%%%%%%%%%%%%%%%%%%%%%%%%%%%%%%%%%%%%%%%%%%%%%%%%%%%%%%%%%
\section{CHAPTER 9}

\textit{Postbases}
\begin{description}[nolistsep, noitemsep]
\item \textbf{$\sim_\text{sf}$--(g)kaq}: one that has V-ed or been V-ed
\item \textbf{$\sim_\text{sf}$--(g)kau}: to have V-ed or been V-ed
\item \textbf{$\sim_\text{f}$ni}: to say that one V-s
\item \textbf{+(te)ste}: to have, make, or let one V
\end{description}

\bigskip

\textit{Third Person Subject Interrogatives for Intransitives}

\begin{tabular}{ l l }
\textbf{$\sim_\text{f}$(g/t)a} & he, she, it\\
\textbf{$\sim_\text{f}$(g/t)at} & they$_\text{\textsc{pl}}$ \\
\textbf{$\sim_\text{f}$(g/t)ak} & they$_\text{\textsc{2}}$ \\
\end{tabular}

\bigskip

\textit{Third Person Subject Interrogatives for Transitives}

\begin{tabular}{ l l }
\textbf{$\sim_\text{f}$(g/t)agu} & he, she, it to him, her, it\\
\textbf{$\sim_\text{f}$(g/t)aki} & he, she, it to them$_\text{\textsc{pl}}$ \\
\textbf{$\sim_\text{f}$(g/t)akek} & he, she, it to them$_\text{\textsc{2}}$ \\
\textbf{$\sim_\text{f}$(g/t)anga} & he, she, it to me \\
\textbf{$\sim_\text{f}$(g/t)aten} & he, she, it to you
\end{tabular}

%%%%%%%%%%%%%%%%%%%%%%%%%%%%%%%%%%%%%%%%%%%%%%%%%%%%%%%%%%%%%%%%%%%%%%%%%
%                                CHAPTER 10                             %
%  2nd Person Subject Optative, Nonsingular 1st Person Subject Optative %
%%%%%%%%%%%%%%%%%%%%%%%%%%%%%%%%%%%%%%%%%%%%%%%%%%%%%%%%%%%%%%%%%%%%%%%%%
\section{CHAPTER 10}

\textit{Postbases}
\begin{description}[nolistsep, noitemsep]
\item \textbf{$\sim_\text{sf}$:(e)sqe}: to ask or tell one to V
\item \textbf{+(te)sug}: to want one to V
\end{description}

\bigskip

\textit{Second Person Subject Optative for Transitives}

\begin{tabular}{ l l }
\textbf{$\varnothing$, +gi, $\sim_\text{f}$i, +n, $\sim_\text{sf}$i, :a} & (you$_1$), V \\
\textbf{$\sim_\text{f}$(i)gu, $\sim_\text{sf}$--ggu} & (you$_1$), V him, her, it \\
\textbf{$\sim_\text{f}$(i)ki} & (you$_1$), V them$_\text{\textsc{pl}}$ \\
\textbf{$\sim_\text{f}$(i)kek} & (you$_1$), V them$_\text{\textsc{2}}$ \\
\textbf{$\sim_\text{f}$(i)nga} & (you$_1$), V me \\
\textbf{$\sim_\text{f}$(i)kut} & (you$_1$), V us$_\text{\textsc{pl}}$ \\
\textbf{@(i)tek} & (you$_\text{\textsc{pl/2}}$), V \\
\textbf{$\sim_\text{sf}$:(e)lta} & let's$_\text{\textsc{pl}}$ V \\
\textbf{$\sim_\text{sf}$:(e)ltung} & let's$_\text{\textsc{2}}$ V
\end{tabular}

\bigskip

\textit{Second Person Subject Negative Optative}

\begin{tabular}{ l l }
\textbf{--fqaavek} & (you$_1$), don't V \\
\textbf{--fqaaftek} & (you$_2$), don't V \\
\textbf{--fqaafsi} & (you$_\text{\textsc{pl}}$), don't V \\
\textbf{--fqaan} & (you$_\text{\textsc{1/pl/2}}$), don't V it \\
\textbf{--fqiita} & (you$_\text{\textsc{1/pl/2}}$), don't V them \\
\textbf{--fqaagkenka} & (you$_\text{\textsc{1/pl/2}}$), don't V them$_2$ \\
\textbf{--fqaama} & (you$_\text{\textsc{1/pl/2}}$), don't V me \\
\textbf{--fqaamta} & (you$_\text{\textsc{1/pl/2}}$), don't V us \\
\textbf{--fqaamtung} & (you$_\text{\textsc{1/pl/2}}$), don't V us$_2$
\end{tabular}


%%%%%%%%%%%%%%%%%%%%%%
%     CHAPTER 11     %
%  Participial Mood  %
%%%%%%%%%%%%%%%%%%%%%%
\section{CHAPTER 11}

\textit{Postbases}
\begin{description}[nolistsep, noitemsep]
\item \textbf{--ligh}: to provide with N; to put N in
\item \textbf{@$\sim$:(i/u)ma}: to have V-ed or to have been V-ed; to evidently have V-ed
\item \textbf{$\sim$:(u)te}: to V with, for or to object; to V object along with oneself; to V each other
\item \textbf{ti-}: to speak the language of N
\item \textbf{@--lghii}: one that is V-ing
\item \textbf{@ngugh*}: one that is V-ing (for descriptive and negative bases ending in \textit{te})
\item \textbf{$\sim_\text{sf}$--(g)ke/(g)ka}: one that the possessor is V-ing
\end{description}

\bigskip

\textsc{[v][intr\_ptcp]}: \textbf{@--lghii, @ngugh*}

\textsc{[v][trns\_ptcp]}: \textbf{$\sim_\text{sf}$--(g)ke/(g)ka}

%%%%%%%%%%%%%%%%%%%%%%%%%%%%%%%%%%%%%%%%%%%%%%%%%%%%%%%%%%
%                    CHAPTER 12                          %
%  Postural Roots, Quantifier-Qualifier, Contractions    %
%%%%%%%%%%%%%%%%%%%%%%%%%%%%%%%%%%%%%%%%%%%%%%%%%%%%%%%%%%
\section{CHAPTER 12}

\textit{Postbases}
\begin{description}[nolistsep, noitemsep]
\item \textbf{$\sim$--kayuk}: one who is able to V
\item \textbf{$\sim$--kayugu}: to be able to V
\item \textbf{+nga}: to be in the Postural Root posture or V state
\item \textbf{+te}: to get or put into the Postural Root posture or V state
\end{description}

\bigskip

\textit{Quantifier-Qualifier Constructions}

\textsc{[quant\_qual][1pl]}: \textbf{:emta}

\textsc{[quant\_qual][1du]}: \textbf{:emtung}

\bigskip

\textit{Note}: In these constructions, the postural root is paired with the possessed relative case endings unless otherwise listed above.

\bigskip

Additional postbases associated with the quantifier-qualifier constructions are listed below:
\begin{description}[nolistsep, noitemsep]
\item \textbf{@$\sim$:(i/u)magh}: while V-ing
\item \textbf{+tuumagh}: together with one's N(s)
\end{description}

\bigskip

\ldots and the obsolete base \textit{\textbf{ete}}.

%%%%%%%%%%%%%%%%%%%%%%%%%%%%%%%%%%%%%%%%%%%%%
%                   CHAPTER 13              %
%  Emotional Roots; Precessive, Concessive  %
%%%%%%%%%%%%%%%%%%%%%%%%%%%%%%%%%%%%%%%%%%%%%
\section{CHAPTER 13}

\textit{Postbases}
\begin{description}[nolistsep, noitemsep]
\item \textbf{@$\sim_\text{f}$miqe}: to make, compel, or force one to V
\item \textbf{$\sim_\text{sf}\text{-}_\text{w}$:(e)nkut}: N and company
\item \textbf{$\sim_\text{sf}\text{-}_\text{w}$:(e)nkuk}: N and partner
\item \textbf{$\sim$--ke} or \textbf{$\sim_\text{f}$yuke}: to feel Emotional Root toward object
\item \textbf{$\sim$--ketagh}: to tend to often feel Emotional Root
\item \textbf{$\sim$--ketaq}: one who tends to often feel Emotional Root
\item \textbf{@$\sim_\text{f}$nagh}: to tend to cause one to feel Emotional Root
\item \textbf{@$\sim_\text{f}$naq}: something that causes one to feel Emotional Root
\item \textbf{@$\sim_\text{f}$yug}: to feel Emotional Root
\end{description}

\bigskip

\textsc{[v][prec]}: \textbf{@$\sim_\text{f}$vagilg(a)}

\bigskip

\textsc{[v][conc]}: \textbf{--ghnga(agh)}

\textit{Note}: It may not be necessary to include the \textsc{[v]} marker with the \textit{connective} moods.

%%%%%%%%%%%%%%%%%%%%%%%%%%%%%%%%%%%%%%%%%%%%%%%%%%%%%%%%%
%                        CHAPTER 14                     %
%  Consequential I and II; Conditional; Contemporative  %
%%%%%%%%%%%%%%%%%%%%%%%%%%%%%%%%%%%%%%%%%%%%%%%%%%%%%%%%%

\section{CHAPTER 14}

\textit{Postbases}
\begin{description}[nolistsep, noitemsep]
\item \textbf{@$\sim_\text{f}$--ragkiigh}: to V quickly
\item \textbf{@$\sim$:(ng)igate}: to never V; to not be V-ing
\end{description}

\bigskip

\textit{Enclitics}
\begin{description}[nolistsep, noitemsep]
\item \textbf{+llu}: and; also
\item \textbf{+nguq}: it is said
\end{description}

\bigskip

\textsc{[v][cnsqi]}: \textbf{@$\sim_\text{f}$y(a)} or \textbf{$\sim_\text{f}$ng(a)}: when one V-ed

\textsc{[v][cnsqii]}: \textbf{$\sim$(g)aqng(a)}: while one was V-ing

\bigskip

\textsc{[v][cond]}: \textbf{@$\sim_\text{sf}$--(g)k(u)}: if or when one V-s

\bigskip

\textsc{[v][cont]}: \textbf{@--negh}: whenever one V-s

%%%%%%%%%%%%%%%%%%%
%    CHAPTER 15   %
%  Subordinative  %
%%%%%%%%%%%%%%%%%%%

\section{CHAPTER 15}

\textit{Postbases}
\begin{description}[nolistsep, noitemsep]
\item \textbf{$\sim$i} or \textbf{$\sim$:(u)te}: to V something
\item \textbf{$\sim$:(ng)ite}: to not have N; to lack N
\item \textbf{@$\sim_\text{f}$naanghite}: to not V in the future; won't V
\item \textbf{@$\sim_\text{f}$na}: in order to V
\end{description}

\bigskip

\textsc{[v][sub]}: \textbf{@$\sim_\text{f}$lu} or \textbf{+na}: V-ing

%%%%%%%%%%%%%%%%%%%%%%%%%%%%%%%%%%%%%%%
%            CHAPTER 16               %
%  Demonstratives; Personal Pronouns  %
%%%%%%%%%%%%%%%%%%%%%%%%%%%%%%%%%%%%%%%

\section{CHAPTER 16}

See Section 16.2.4 on p.109 for inflection of demonstratives.

%%%%%%%%%%%%%%%%%%%%%%%%%%%%%%%%%%%%%%%%%%%%%%%%%%%%%%%%%%%
%                         CHAPTER 17                      %
%  1/3SBJ Optative; Future Optative; Participial-Oblique  %
%%%%%%%%%%%%%%%%%%%%%%%%%%%%%%%%%%%%%%%%%%%%%%%%%%%%%%%%%%%

\section{CHAPTER 17}

\textit{Postbases}
\begin{description}[nolistsep, noitemsep]
\item \textbf{$\sim$--ke}: to have as one's N
\item \textbf{$\sim$:(u)te} or \textbf{$\sim$:(u)ta}: condition with respect to V-ing
\item \textbf{+te} or \textbf{+qe}: to act on so as to cause to V
\item \textbf{@$\sim_\text{f}$vik}: place to V
\item \textbf{@$\sim_\text{f}$yagh}: to go V-ing; to V in vain
\item \textbf{--pigte}: to be very V; to V very much
\item \textbf{--ghhagh*}: little N; dear N
\end{description}

\bigskip

\textit{Selected First/Third Person Subject Optative}

\begin{tabular}{ l l }
\textbf{@$\sim_\text{f}$li} & let him V, may he V \\
\textbf{@$\sim_\text{f}$ligu} & let him V him, may he V him \\
\textbf{@$\sim_\text{f}$liki} & let him V them, may he V them \\
\textbf{@$\sim_\text{f}$linga} & let him V me, may he V me \\
\textbf{@$\sim_\text{f}$liten} & let him V you, may he V you \\
\textbf{@$\sim_\text{f}$langa} & let me V, may I V \\
\textbf{@$\sim_\text{f}$lakun} & let me V it, may I V it \\
\textbf{@$\sim_\text{f}$langi} & let me V them, may I V them \\
\textbf{@$\sim_\text{f}$laken} & let me V you, may I V you
\end{tabular}

\bigskip

\textsc{[v][intr\_opt][fut]}: \textbf{@--lgha/$\sim_\text{sf}$--ghha/$\sim_\text{sf}$--gga}

\textsc{[v][trns\_opt][fut]}: \textbf{@$\sim_\text{f}$na}

\bigskip

\textsc{[v][[opt][fut][neg]}: \textbf{@$\sim_\text{f}$yaquna}

\bigskip

\textsc{[v][intr\_ptcp-obl]}: \textbf{@$\sim_\text{f}$yalghii}

\textsc{[v][trns\_ptcp-obl]}: \textbf{@$\sim_\text{f}$yaqe}

%%%%%%%%%%%%%%%%%%%%%%%%%%%%%%%%%%%%%%%%%%%%%%%%%%%%%%%%%%%
%                         CHAPTER 18                      %
%  Positional Bases  %
%%%%%%%%%%%%%%%%%%%%%%%%%%%%%%%%%%%%%%%%%%%%%%%%%%%%%%%%%%%

\section{CHAPTER 18}

\textit{Postbases}
\begin{description}[nolistsep, noitemsep]
\item \textbf{@$\sim$laatagh}: to V anyway, despite everything, to still V
\item \textbf{@$\sim_\text{f}$nanigh}: to cease V-ing; to stop V-ing
\item \textbf{--neq}: the act or activity of V-ing
\item \textbf{qugh}: to V completely
\item \textbf{$\sim_\text{f}$yaghtugh}: to be about to V; to go somewhere to V
\end{description}

\bigskip

See p.126-127 for list of positional bases and their inflectional endings.

\bigskip

\textsc{[v][vol-of-fear]}: \textbf{nayuka}

\bigskip

\textsc{[voc]}: \textbf{--y} with final-vowel lengthening

\end{document}
