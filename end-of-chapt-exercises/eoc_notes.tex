\documentclass{article}

\usepackage{color}
\usepackage{enumitem}

\begin{document}

\begin{itemize}
\renewcommand\labelitemi{$\cdot$}

\item 4: \textit{\textbf{nuyaqatakestaaghhaaguq}} will resolve with Ch.18, when we account for postbases that can attach to inflected nouns.

\item 4: \textcolor{magenta}{nanevgaq}

\item 4: \textcolor{magenta}{nengyugka} 

\item 4: One of the consonants \textit{wh}/\textit{ll} in {\textit{\textbf{aawhllaget}}} should undouble.

\item 5: \textcolor{magenta}{aaken} 

\item 5: The \textit{gh} should have dropped in \textit{\textbf{naayghaghmeng}}, since it is not marked as strong.

\item 6: The \textit{gh} should have dropped in \textit{\textbf{sivuqaghmeng}}, since it is not marked as strong.

\item 7: \textit{\textbf{apeghtughistegpekliilleqii}} will resolve with Ch.18, when we account for postbases that can attach to inflected nouns.

\item 7: \textcolor{magenta}{nanevgam}

\item 8: The \textit{gh} should have dropped in \textit{\textbf{sivuqaghmun}}, since it is not marked as strong.

\item 8: The citation form is given as \textit{alquutagh} for the base in \textit{\textbf{alquughtaten}}. 

\item 8: \textit{\textbf{sangavek}} will resolve with the implementation of the consequential mood. 

\item 8: \textcolor{magenta}{nagatestaak}

\item 8: \textcolor{magenta}{quyillgaat}

\item 8: If the analysis for \textit{\textbf{aghveliighsiin}} is \textit{aghvegh-–liigh[TO-COOK-OR-PREPARE-N][N→V][V][Intrg][2Sg]}, then the given surface form has an extra \textit{i}.

\item 8: \textit{\textbf{angyangllaghyugsiin}} has an extra \textit{i}?

\item 8: \textit{\textbf{qikmighllalguziin}} has an extra \textit{i}?
%
Unless I'm missing something.

\item 8: If the analysis of \textcolor{magenta}{neqeka} is \textit{neqe[N][Abs][1SgPoss][SgPosd]}, how does \textit{e}-dropping work here?

\item 8: There may be a possible typo here in \textit{\textbf{laalighsigu}}, since the citation form is \textit{laalighte}.
%
Final-\textit{e} shouldn't drop from this long base, yes?

\item 9: If the analysis of \textcolor{magenta}{ivaghta} is \textit{ivagh[V][Intrg][3Sg]}, this shouldn't be possible, since \textit{ivagh} is transitive only, and the interrogative form is for intransitives.

\item 9: \textcolor{magenta}{ighneghaan}

\item 9: \textit{neghegkaawaa} and \textit{aghviqaawa} should resolve when the postbase \textit{@*$\sim_\text{sf}$--(g*)kaq} is implemented correctly.

\item 9: \textcolor{magenta}{kitumun}

\item 9: If the analysis of \textcolor{magenta}{naallghutestesiki} is \textit{naallghute(te)ste[V][Intrg][2Sg][3Pl]}, why does the \textit{z} become an \textit{s}?

\item 9: \textcolor{magenta}{qergesengi}

\end{itemize}

\end{document}
