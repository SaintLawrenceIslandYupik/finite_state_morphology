\documentclass{article}

\usepackage{color}
\usepackage{enumitem}

\begin{document}

\begin{itemize}
\renewcommand\labelitemi{$\cdot$}

\item \textit{\textbf{nuyaqatakestaaghhaaguq}} will resolve with Ch.18, when we account for postbases that can attach to inflected nouns.

\item \textcolor{\magenta}{nanevgaq}

\item \textcolor{\magenta}{nengyugka} 

\item One of the consonants \textit{wh}/\textit{ll} in {\textit{\textbf{aawhllaget}}} should undouble.

\item \textcolor{\magenta}{aaken} 

\item \textcolor{\magenta}{neghellequnga} uses the alternative base form for \textit{negh}.

\item The \textit{gh} should have dropped in \textit{\textbf{naayghaghmeng}}, since it is not marked as strong.

\item The \textit{gh} should have dropped in \textit{\textbf{sivuqaghmeng}}, since it is not marked as strong.

\item \textcolor{\magenta}{neghellequt} uses the alternative base form for \textit{negh}.

\item \textit{\textbf{apeghtughistegpekliilleqii}} will resolve with Ch.18, when we account for postbases that can attach to inflected nouns.

\item \textcolor{\magenta}{nanevgam}

\item \textcolor{\magenta}{neghenghitaqa} uses the alternative base form for \textit{negh}.

\item \textcolor{\magenta}{neghelleqesta} uses the alternative base form for \textit{negh}.

\item The \textit{gh} should have dropped in \textit{\textbf{sivuqaghmun}}, since it is not marked as strong.

\item The citation form is given as \textit{alquutagh}} for the base in \textit{\textbf{alquughtaten}}. 

\item \textit{\textbf{sangavek}} will resolve with the implementation of the consequential mood. 

\item \textcolor{\magenta}{nagatestaak}

\item \textcolor{\magenta}{quyillgaat}

\item If the decomposition for \textit{\textbf{aghveliighsiin} is \textit{aghvegh-–liigh[TO-COOK-OR-PREPARE-N][N→V][V][Intrg][2Sg]}, then the given surface form has an extra \textit{i}.

\item \textit{\textbf{angyangllaghyugsiin}} has an extra \textit{i}?

\item \textit{\textbf{qikmighllalguziin}} has an extra \textit{i}?

\item If the decomposition of \textcolor{\magenta}{neqeka} is \textit{neqe[N][Abs][1SgPoss][SgPosd]}, how does \textit{e}-dropping work here?

\item There may be a possible typo here in \textit{\textbf{laalighsigu}}, since the citation form is \textit{laalighte}} \ldots

\end{itemize}

\end{document}
