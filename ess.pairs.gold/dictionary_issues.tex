\documentclass{article}

\usepackage{amsmath}
\usepackage{enumitem}
\usepackage{setspacing}
\onehalfspacing

\begin{document}

%%%%%%%%%%%%%%%%%%%%%%%%%%%%%%%%%%%
\section{General Dictionary Issues}
%%%%%%%%%%%%%%%%%%%%%%%%%%%%%%%%%%%

\begin{enumerate}

\item What does \% mean? e.g. \textbf{\%(e)nkuk} which is defined in Jacobson (2001) as \textbf{$\sim_\text{sf}$-w:(e)nkuk}

\item What does ? mean? e.g. \textbf{?aq}

\item Some particles are defined with a (n), e.g. \textbf{qakuqwaagu(n)}

\item Some nouns are defined with a (t), e.g. \textbf{Atqalghhaaghmii(t)}

\end{enumerate}


%%%%%%%%%%%%%%%%%%%%%%%%%%%%%%%%%%
\section{Postbase-Specific Issues}
%%%%%%%%%%%%%%%%%%%%%%%%%%%%%%%%%%

\begin{enumerate}

\item Is \textbf{--ghllag} a [N$\rightarrow$V] postbase?

\item Some postbases are limited in terms of what types of nouns they may affix to, e.g. vowel-ending, animate, etc. Is this something we should account for?

\item Is \textbf{?aq} missing morphophonological symbols? Is there a pattern to the inflected examples that are given?

\item Is the postbase \textbf{$\sim_\text{sf}$:(e)sqe} as Jacobson (2001) claims or is the underlying \textit{e} an \textit{i}?

\item The string \textit{kiiw-–ghhagh*[N$\rightarrow$N][N][Abs][Unpd][Sg]} should yield \textbf{kiiwhaq}. Is final consonant drop assimilation taking place? The explanation in Jacobson (2001) in Section 4.2.2 does not cover cases involving \textit{ghh}.

\item \textbf{nengpagtuq} and \textbf{nengsugpagtuq} are irregular forms of \textit{nenge-@$_\text{1}$-ghpagte}...

\item \textit{nuugte-@$_\text{1}$-ghpagte}... yields \textbf{nuugpagte}..., dropping base-final \textit{te} and the final consonant, which contradicts earlier observations, e.g. \textit{laalighte[V][Intr][Cond][3Sg]

\end{enumerate}


%%%%%%%%%%%%%%%%%%%%%%
\section{Other Issues}
%%%%%%%%%%%%%%%%%%%%%%

\begin{enumerate}

\end{enumerate}

\end{document}
