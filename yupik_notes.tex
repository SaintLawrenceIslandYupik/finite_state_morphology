\documentclass{article}

\setlength{\parindent}{0pt}
\usepackage{amsmath}
\usepackage{ulem}

\begin{document}

\section{}

%%%%%%%%%%%%%%%%%%%%%%%%%%%%%%%%%%%%%%%%%%
%               CHAPTER 2                %
%   Bases, Unpossessed Plural and Dual   %
%%%%%%%%%%%%%%%%%%%%%%%%%%%%%%%%%%%%%%%%%%
\section{CHAPTER 2: Words and Bases; Formation of Unpossessed Plural and Dual
Absolutive Case; Conventional Duals}

%%%%%%%%%%%%%%%%%%%%%%%%%%%%%
%   Citation to Base Form   %
%%%%%%%%%%%%%%%%%%%%%%%%%%%%%
\subsection{Citation Form to Base Form (Nouns)}

\begin{enumerate}
\item A noun that ends in a full vowel in the citation form will have a base form just like the citation form.
\item Some nouns ending in \textit{-a} have bases ending in \textit{-e}. These are marked with a superscript \textit{e}.
\begin{description}
\item Nearly all nouns ending in \textit{-ta} have bases ending in \textit{-te}, so the superscript \textit{-e} is not considered necessary. The few exceptions where the citation and base forms both end in \textit{-ta} are marked.
\end{description}
\item Nouns that end in \textit{-n} have bases that end in \textit{-te} instead.
\begin{description}
\item This suggests that any base form ending in \textit{-te} may have a citation form ending in \textit{n} or \textit{ta}. If there is a consonant before \textit{-te} in the base however, the citation form can only end in \textit{-ta}.
\end{description}
\item Nouns that end in \textit{-q}, \textit{-k}, \textit{-qw}, or \textit{-kw} have bases that end in \textit{-gh}, \textit{-g}, \textit{-ghw}, and \textit{-w} respectively.
\item \textsc{exception}: The word \textit{yuuk} has the base \textit{yug}.
\end{enumerate}

%%%%%%%%%%%%%%%%%%%%%%%%%%%%%%%%%%%%%%%%%%%%%%
%   Unpossessed Absolutive Plural and Dual   %
%%%%%%%%%%%%%%%%%%%%%%%%%%%%%%%%%%%%%%%%%%%%%%
\subsection{Unpossessed Absolutive Plurals and Duals}

\textsc{plural}: \textbf{${\sim}_\text{sf}\text{-}_\text{w}$:(e)t}

\noindent \textsc{dual}: \textbf{${\sim}_\text{sf}\text{-}_\text{w}$:(e)k}

\begin{enumerate}
\item \textbf{${\sim}_\text{sf}$}: Drop \textit{semi-final} \textbf{e} from the base if it exists (This is an `\textit{e}' that is followed by another letter).
\begin{description}
\item If dropping semi-final `\textit{e}' results in a cluster of three consonants within the word or two consonants at the beginning of the word, `\textit{e}' is \textsc{not} dropped.
\item If the semi-final `\textit{e}' is dropped and the base has only one full vowel preceding the `\textit{e}', that full vowel is lengthened and this is called \textbf{e} \textit{hopping}.
\end{description}
\item \textbf{$\text{-}_\text{w}$}: \textit{Weak} consonants are dropped from the base. These \underline{exclude} `\textit{g}', `\textit{ghw}' and `\textit{w}', but `\textit{gh}' may be either weak or strong. A `\textit{gh}' that is preceded by `\textit{e}' or two full vowels is invariably strong, while a `\textit{gh}' that is preceded by one full vowel is most likely weak. Unexpectedly strong `\textit{gh}'s are marked with an * in vocabulary lists.
\item \textbf{(e)t} \textit{or} \textbf{(e)k}: Suffix this `\textit{e}' if the base at this point ends in a strong consonant. Whether this `\textit{e}' is added, always suffix `\textit{t}' or `\textit{k}' at the end.
\item \textbf{:}: \textit{Uvular dropping} is indicated by the colon `:'. If the uvular `\textit{gh}' appears between two short vowels after the above operations, the first of which is a full vowel, the `\textit{gh}' drops, and the `\textit{e}' changes to match the full vowel it is now next to.
\end{enumerate}


%%%%%%%%%%%%%%%%%%%%%%%%%%%%%%%%%%%%%%%%%%%%%%%%%%%%%%%%%%%%%
%                         CHAPTER 3                         %
%   Intransitive Indicative, Unpossessed Ablative-Modalis   %
%%%%%%%%%%%%%%%%%%%%%%%%%%%%%%%%%%%%%%%%%%%%%%%%%%%%%%%%%%%%%
\section{CHAPTER 3: Intransitive Indicative Mood; Unpossessed Ablative-Modalis Case}

%%%%%%%%%%%%%%%%%%%%%%%%%%%%%%%
%   Intransitive Indicative   %
%%%%%%%%%%%%%%%%%%%%%%%%%%%%%%%
\subsection{Intransitive Indicative}

\noindent Rule: \textbf{${\sim}_\text{f}$(g/t)u-}

\begin{enumerate}
\item \textbf{${\sim}_\text{f}$}: Drop \textit{final} \textbf{e} from the base if it exists.
\item \textbf{(g/t)}: Suffix `\textit{g}' if the base ends in a double vowel, but suffix `\textit{t}' if the base ends in a consonant.
\item \textbf{u}: Suffix `\textit{u}'.
\item \textbf{-}: The hyphen indicates the expectation of another morpheme, in this case, a morpheme indicating person/number, the possibilities of which are given below.
\end{enumerate}

\textbf{Person/Number Markings for Intransitive Indicative}

\begin{tabular}{ l l }
\textbf{-q} & he, she, it \\ 
\textbf{-t} & they$_\text{\textsc{pl}}$ \\  
\textbf{-k} & they$_\text{2}$ \\
\textbf{-nga} & I \\
\textbf{-kut} & we$_\text{\textsc{pl}}$ \\  
\textbf{-kung} & we$_\text{2}$ \\  
\textbf{-ten} & you$_\text{1}$ \\
\textbf{-si} & you$_\text{\textsc{pl}}$ \\
\textbf{-tek} & you$_\text{2}$
\end{tabular}

\vspace{12pt}

Some verb bases are presented with angled brackets around a final `\textit{e}', e.g. \textbf{negh$\langle$e$\rangle$-} and \textbf{megh$\langle$e$\rangle$-}.
%
This indicates that with some endings, these verbs behave as \textbf{negh-} and \textbf{megh-}, and with others as \textbf{neghe-} and \textbf{meghe-}.

\vspace{12pt}

When forming the intransitive indicative of verb bases that end in a single vowel, \textit{vowel dominance} prevents unlike vowels from grouping together.
%
Underlying `\textit{ai}', `\textit{ia}', and `\textit{iu}' become `\textit{ii}', and underlying `\textit{au}' and `\textit{ua}' become `\textit{aa}'.
%
Only underlying `\textit{uu}' will result in a surface `\textit{uu}'.

%%%%%%%%%%%%%%%%%%%%%%%%
%   Ablative Modalis   %
%%%%%%%%%%%%%%%%%%%%%%%%
\subsection{Ablative Modalis}

\textsc{unpossessed singular}: \textbf{$\sim_\text{f}$-$_\text{w}$meng}

\vspace{12pt}

Yupik intransitive sentences can have an object of sorts but nothing about this object is indicated in the ending of the verb.
%
This object is called an \textit{indefinite} object, and a noun serving as an indefinite object is put in the \textit{\textbf{ablative-modalis}} case.


%%%%%%%%%%%%%%%%%%%%%%%%%%%%%%%%%%%%%%%%%%%%%%%%%%%%%%%%%%%%%%
%                         CHAPTER 4                          %
%   1st and 2nd Person Possessor Possessed Absolutive Case   %
%%%%%%%%%%%%%%%%%%%%%%%%%%%%%%%%%%%%%%%%%%%%%%%%%%%%%%%%%%%%%%
\section{CHAPTER 4: 1\textsuperscript{st} and 2\textsuperscript{nd} Person Possessor Possessed Absolutive Case}

%%%%%%%%%%%%%%%%%%%%%%%%%%%
%   Chapter 4 Postbases   %
%%%%%%%%%%%%%%%%%%%%%%%%%%%
\subsection{Discussion of Postbases}

Postbases fit between a base and an ending, and there may be more than one postbase in a word.
%
They can be approximately grouped as follows:

\begin{description}
\item \textit{Noun-elaborating}: Affix to noun bases and yield nouns
\item \textit{Verb-elaborating}: Affix to verb bases and yield verbs
\item \textit{Nominalizing}: Affix to verb bases and yield nouns
\item \textit{Verbalizing}: Affix to noun bases and yield verbs
\end{description}

Examples:

\begin{description}
\item \textbf{-ghllak}: big N, lots of N
\item \textbf{-ghrugllak}: big or huge N
\item \textbf{$\sim$:(ng)u}: to be N
\item \textbf{-lek}: one having N
\end{description}

If a postbase or ending starts with a velar, uvular fricative or stop, the dropped final consonant of a base always influences the initial segment of that postbase or ending.

%%%%%%%%%%%%%%%%%%%%%%%%%%%%%%%%%%%%%%%%%%%%%%%%%%%%%%%%%%%%%%
%   1st and 2nd Person Possessor Possessed Absolutive Case   %
%%%%%%%%%%%%%%%%%%%%%%%%%%%%%%%%%%%%%%%%%%%%%%%%%%%%%%%%%%%%%%
\subsection{1\textsuperscript{st} and 2\textsuperscript{nd} Person Possessor Possessed Absolutive}

\textbf{Endings for Singular Possessor}

\begin{tabular}{ l l }
\textbf{$\sim$-ka} & my N \\ 
\textbf{${\sim}_\text{sf}\text{-}_\text{w}$:(e)nka} & my N$_\text{\textsc{pl}}$ \\  
\textbf{${\sim}_\text{sf}\text{-}_\text{w}$:(e)gka} & my N$_\text{\textsc{2}}$ \\  
\textbf{${\sim}_\text{sf}\text{-}_\text{w}$:(e)n} & your$_1$ N \\  
\textbf{+ten} & your$_1$ N$_\text{\textsc{pl}}$ \\
\textbf{${\sim}_\text{sf}\text{-}_\text{w}$:(e)gken} & your$_1$ N$_\text{\textsc{2}}$ \\  
\end{tabular}

\vspace{12pt}

The first ending, \textit{-ka}, drops final and semi-final `\textit{e}', and hops them if possible.
%
Otherwise it attaches like a postbase, and drops all final consonants.
%
If the dropped final consonant is \underline{uvular}, the possessor ending becomes \textit{-qa}.
%
If the dropped final consonant is \underline{rounded}, the ending becomes \textit{-kwa} or \textit{-qwa}.

\vspace{12pt}

The fifth ending, \textit{$+$ten}, retains all final consonants.
%
The plus sign indicates that no alteration is made to the base, and this sign will be used in citations to distinguish a suffix which does not
alter a base in any way from the word.

\vspace{12pt}

\noindent \textbf{Endings for Plural Possessor}

\begin{tabular}{ l l }
\textbf{+put/+vut} & our N \\ 
\textbf{-put} & our N$_\text{\textsc{pl}}$ \\ 
\textbf{+si/+zi} & your$_\text{\textsc{pl}}$ N \\ 
\textbf{-si} & your$_\text{\textsc{pl}}$ N$_\text{\textsc{pl}}$ \\ 
\end{tabular}

\vspace{12pt}

The forms \textit{$+$put} and \textit{$+$si} are used with bases that end in a consonant, while \textit{$+$vut} and \textit{$+$zi} are used with bases that end in a vowel.


%%%%%%%%%%%%%%%%%%%%%%%%%%%%%%%%%%%%%%%%%%%%%%%%%%%%%%%%%%%%%%%%%%
%                           CHAPTER 5                            %
%   Unpossessed Localis, Terminalis, Vialis, and Equalis Cases   %
%%%%%%%%%%%%%%%%%%%%%%%%%%%%%%%%%%%%%%%%%%%%%%%%%%%%%%%%%%%%%%%%%%
\section{CHAPTER 5: Unpossessed Localis, Terminalis, Vialis, and Equalis Cases}

%%%%%%%%%%%%%%%%%%%%%%%%%%%
%   Chapter 5 Postbases   %
%%%%%%%%%%%%%%%%%%%%%%%%%%%
\subsection{Some Verb-Elaborating Postbases}

\textbf{-(g)aqe-}

This postbase indicates a present time ongoing action, and may also be used to represent regular or repeated action. 
%
The (g) is used with bases that end in two vowels.
%
Observe that in Yupik, the \textit{unmarked} verb form implies past-time action.

\vspace{12pt}

\textbf{@lleqe}

This postbase is used for future action, \textit{will V}.

The symbol @ indicates that \textit{-te} on a verb base will be modified in some way.
%
Here, it means that `@lleqe-' drops all \textit{-te}'s.

\vspace{12pt}

\textbf{@$\sim_\text{f}$naqe-}

This postbase means \textit{to be about to}, or \textit{going to V}, and is used for future action more imminent than that expressed by the previous postbase.

It is one of a number of postbases that start with `\textit{n}' that drop base-final \textit{-te}.
%
If \textit{-te} followed a vowel in the base, the `\textit{n}' of the postbase becomes voiceless, e.g. \textit{\textbf{kaate-} | to arrive} $\rightarrow$ \textbf{\textit{kaa\underline{nn}aquq}}.
%
If \textit{-te} followed a consonant however, the resulting cluster of consonant and `\textit{n} becomes voiceless as a whole, e.g. \textit{\textbf{ingaghte-} | to lie down} $\rightarrow$ \textit{\textbf{inga\underline{ghhn}aquq}}.

\vspace{12pt}

\textbf{@-fyug-}

This postbase means \textit{to want to V}.
%
Since voiceless `\textit{y}' is `\textit{s}', the `\textit{y}' of this postbase becomes `\textit{s}' after a stop and combines with `\textit{t} to yield `\textit{s}'.

%%%%%%%%%%%%%%%%%%%%%%%%%%%%%%%%%%%
%   Localis and Terminalis Case   %
%%%%%%%%%%%%%%%%%%%%%%%%%%%%%%%%%%%
\subsection{Localis and Terminalis Cases}

\textsc{unpossessed singular}: \textbf{$\sim_\text{f}$-$_\text{w}$mi}

The \textit{localis} case indicates the place \underline{at} which an action occurs.

\vspace{12pt}

\textsc{unpossessed singular}: \textbf{$\sim_\text{f}$-$_\text{w}$mun}

The \textit{terminalis} case indicates the place \underline{to} which an action occurs.

\vspace{12pt}

The \textsc{plurals} of these cases have `\textit{n}' in place of `\textit{m}', but final `\textit{e}' is dropped only from ???longer bases.

%%%%%%%%%%%%%%%%%%%%
%   Vialis Case   %
%%%%%%%%%%%%%%%%%%%%
\subsection{Vialis Case}

\textsc{singular}: \textbf{$\sim_\text{f}$-$_\text{w}$kun}

\textsc{plural}: \textbf{$\sim_\text{sf}$-$_\text{w}$:(e)tgun}

\vspace{12pt}

The \textit{vialis case} indicates place \underline{through} which an action occurs, or place \textit{to} which an action occurs if the motion is downward.
%
When used with a base that ends with a strong `\textit{gh}', the resulting cluster \textit{-ghk} becomes \textit{-ghq}.

\vspace{12pt}

The vialis case has one basic non-orientational use, to indicate \textit{instrument}, e.g. \textit{\textbf{tagiiq angyakun} | he came \underline{by} boat}.

%%%%%%%%%%%%%%%%%%%%
%   Equalis Case   %
%%%%%%%%%%%%%%%%%%%%
\subsection{Equalis Case}

\textsc{singular}: \textbf{$\sim_\text{f}$-$_\text{w}$tun}

\textsc{plural}: \textbf{$\sim_\text{sf}$-$_\text{w}$:(e)stun}

\vspace{12pt}

The \textit{equalis} case only has a non-orientational use, to make a comparison, e.g. \textit{\textbf{neghtuq qikmitun} | he ate \underline{like} a dog}.



\end{document}
  
