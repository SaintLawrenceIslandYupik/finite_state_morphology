\documentclass{article}

\setlength{\parindent}{0pt}
\usepackage{amsmath}

\begin{document}

\section{}

%%%%%%%%%%%%%%%%%
%   CHAPTER 2   %
%%%%%%%%%%%%%%%%%
\section{CHAPTER 2: Words and Bases; Formation of Unpossessed Plural and Dual
Absolutive Case; Conventional Duals}

\subsection{Citation Form to Base Form (Nouns)}

\begin{enumerate}
\item A noun that ends in a full vowel in the citation form will have a base form just like the citation form.
\item Some nouns ending in `\textit{-a}' have bases ending in `\textit{-e}'. These are marked with a superscript `\textit{e}'.
\begin{description}
\item Nearly all nouns ending in `\textit{-ta}' have bases ending in `\textit{-te}', so the superscript `\textit{-e}' is not considered necessary. The few exceptions where the citation and base forms both end in `\textit{-ta}' are marked.
\end{description}
\item Nouns that end in `\textit{-n}' have bases that end in `\textit{-te}' instead.
\begin{description}
\item This suggests that any base form ending in `\textit{-te}' may have a citation form ending in `\textit{n}' or `\textit{ta}'. If there is a consonant before `\textit{-te}' in the base however, the citation form can only end in `\textit{-ta}'.
\end{description}
\item Nouns that end in `\textit{-q}', `\textit{-k}', `\textit{-qw}', or `\textit{-kw}' have bases that end in `\textit{-gh}', `\textit{-g}', `\textit{-ghw}', and `\textit{-w}' respectively.
\item \textsc{exception}: The word `\textit{yuuk}' has the base `\textit{yug}'.
\end{enumerate}

\subsection{Unpossessed Absolutive Plurals and Duals}

\textsc{plurals}: \textbf{${\sim}_\text{sf}\text{-}_\text{w}$:(e)t}

\noindent \textsc{duals}: \textbf{${\sim}_\text{sf}\text{-}_\text{w}$:(e)k}

\begin{enumerate}
\item \textbf{${\sim}_\text{sf}$}: Drop \textit{semi-final} \textbf{e} from the base if it exists (This is an `e' that is followed by another letter).
\begin{description}
\item If dropping semi-final `e' results in a cluster of three consonants within the word or two consonants at the beginning of the word, `e' is \textsc{not} dropped.
\item If the semi-final `e' is dropped and the base has only one full vowel preceding the `e', that full vowel is lengthened and this is called \textbf{e} \textit{hopping}.
\end{description}
\item \textbf{$\text{-}_\text{w}$}: \textit{Weak} consonants are dropped from the base. These \underline{exclude} `\textit{g}', `\textit{ghw}' and `\textit{w}', but `\textit{gh}' may be either weak or strong. A `\textit{gh}' that is preceded by `\textit{e}' or two full vowels is invariably strong, while a `\textit{gh}' that is preceded by one full vowel is most likely weak. Unexpectedly strong `\textit{gh}'s are marked with an * in vocabulary lists.
\item \textbf{(e)t} \textit{or} \textbf{(e)k}: Suffix this `\textit{e}' if the base at this point ends in a strong consonant. Whether this `\textit{e}' is added, always suffix `\textit{t}' or `\textit{k}' at the end.
\item \textbf{:}: \textit{Uvular dropping} is indicated by the colon `:'. If the uvular `\textit{gh}' appears between two short vowels after the above operations, the first of which is a full vowel, the `\textit{gh}' drops, and the `\textit{e}' changes to match the full vowel it is now next to.
\end{enumerate}


%%%%%%%%%%%%%%%%%
%   CHAPTER 3   %
%%%%%%%%%%%%%%%%%
\section{CHAPTER 3: Intransitive Indicative Mood; Unpossessed Ablative-Modalis Case}

\subsection{Intransitive Indicative}

\noindent Rule: \textbf{${\sim}_\text{f}$(g/t)u-}

\begin{enumerate}
\item \textbf{${\sim}_\text{f}$}: Drop \textit{final} \textbf{e} from the base if it exists.
\item \textbf{(g/t)}: Suffix `\textit{g}' if the base ends in a double vowel, but suffix `\textit{t}' if the base ends in a consonant.
\item \textbf{u}: Suffix `\textit{u}'.
\item \textbf{-}: The hyphen indicates the expectation of another morpheme, in this case, a morpheme indicating person/number, the possibilities of which are given below.
\end{enumerate}

\textbf{Person/Number Markings for Intransitive Indicative}

\begin{tabular}{ l l }
\textbf{-q} & he, she, it \\ 
\textbf{-t} & they$_\text{\textsc{pl}}$ \\  
\textbf{-k} & they$_\text{2}$ \\
\textbf{-nga} & I \\
\textbf{-kut} & we$_\text{\textsc{pl}}$ \\  
\textbf{-kung} & we$_\text{2}$ \\  
\textbf{-ten} & you$_\text{1}$ \\
\textbf{-si} & you$_\text{\textsc{pl}}$ \\
\textbf{-tek} & you$_\text{2}$
\end{tabular}

\vspace{12pt}

Furthermore, some verb bases are presented with angled brackets around a final `\textit{e}' such as \textbf{negh$\langle$e$\rangle$-} and \textbf{megh$\langle$e$\rangle$-}.
%
This indicates that with some endings, these verbs behave as \textbf{negh-} and \textbf{megh-}, and with others as \textbf{neghe-} and \textbf{meghe-} respectively.

Lastly, when the intransitive indicative of verb bases ending in a single vowel is formed, \textit{vowel dominance} prevents unlike vowels from grouping together.
%
Particularly, underlying `\textit{ai}', `\textit{ia}', and `\textit{iu}' become `\textit{ii}', and underlying `\textit{au}' and `\textit{ua}' become `\textit{aa}'.
%
Only underlying `\textit{uu}' will result in a surface `\textit{uu}'.

\subsection{Absolutive Modalis}

Yupik intransitive sentences can have an object of sorts but nothing about this object is indicated in the ending of the verb.
%
This object is called an \textit{indefinite} object, and a noun serving as an indefinite object is put in the \textit{\textbf{ablative-modalis}} case.

The unpossesed ablative-modalis singular ending is \textbf{$\sim_\text{f}$-$_\text{w}$meng}.


%%%%%%%%%%%%%%%%%
%   CHAPTER 4   %
%%%%%%%%%%%%%%%%%
\section{CHAPTER 4: 1\textsuperscript{st} and 2\textsuperscript{nd} Person Possessor Possessed Absolutive Case}

\subsection{Discussion of Postbases}

Postbases fit between a base and an ending, and there may be more than one postbase in a word.
%
They can be approximately grouped as follows:

\begin{description}
\item \textit{Noun-elaborating}: Affix to noun bases and yield nouns
\item \textit{Verb-elaborating}: Affix to verb bases and yield verbs
\item \textit{Nominalizing}: Affix to verb bases and yield nouns
\item \textit{Verbalizing}: Affix to noun bases and yield verbs
\end{description}

Examples:

\begin{description}
\item \textbf{-ghllak}: big N, lots of N
\item \textbf{-ghrugllak}: big or huge N
\item \textbf{$\sim$:(ng)u}: to be N
\item \textbf{-lek}: one having N
\end{description}

Again, a dropped final consonant of a base always influences the initial segment of a postbase or ending in this way if that postbase or ending starts with a velar, uvular fricative or stop.

\subsection{1\textsuperscript{st} and 2\textsuperscript{nd} Person Possessor Possessed Absolutive}

\textbf{Endings for Singular Possessor}

\begin{tabular}{ l l }
\textbf{$\sim$-ka} & my N \\ 
\textbf{${\sim}_\text{sf}\text{-}_\text{w}$:(e)nka} & my N$_\text{\textsc{pl}}$ \\  
\textbf{${\sim}_\text{sf}\text{-}_\text{w}$:(e)gka} & my N$_\text{\textsc{2}}$ \\  
\textbf{${\sim}_\text{sf}\text{-}_\text{w}$:(e)n} & your$_1$ N \\  
\textbf{+ten} & your$_1$ N$_\text{\textsc{pl}}$ \\
\textbf{${\sim}_\text{sf}\text{-}_\text{w}$:(e)gken} & your$_1$ N$_\text{\textsc{2}}$ \\  
\end{tabular}

\vspace{12pt}

Note that the first ending, \textit{-ka}, drops final and semi-final e, and hops them if possible.
%
Otherwise it attaches like a postbase, and drops all final consonants.
%
If the dropped final consonant is uvular, then this ending takes the form \textit{-qa} for that word, and if the dropped final consonant is rounded, then this ending takes the form \textit{-kwa} or \textit{-qwa}.

The fifth ending, \textit{$+$ten}, retains all final consonants.
%
The plus sign indicates that no alteration is made to the base, and this sign will be used in citations to distinguish a suffix which does not
alter a base in any way from the word.

\vspace{12pt}

\noindent \textbf{Endings for Plural Possessor}

\begin{tabular}{ l l }
\textbf{+put/$+$vut} & our N \\ 
\textbf{-put} & our N$_\text{\textsc{pl}}$ \\ 
\textbf{+si/$+$zi} & your$_\text{\textsc{pl}}$ N \\ 
\textbf{-si} & your$_\text{\textsc{pl}}$ N$_\text{\textsc{pl}}$ \\ 
\end{tabular}

\vspace{12pt}

The forms \textit{$+$put} and \textit{$+$si} are used with consonant-ending bases, while \textit{$+$vut} and \textit{$+$zi} are used with vowel-ending bases.
\end{document}
