\documentclass{article}

\usepackage{amsmath}

\begin{document}

\section{}

\section{CHAPTER 2: Words and Bases; Formation of Unpossessed Plural and Dual
Absolutive Case; Conventional Duals}

\subsection{Citation Forms to Bases}

\begin{enumerate}
\item A noun that ends in a full vowel in the citation form will have a base form just like the citation form.
\item Some nouns ending in `\textit{-a}' have bases ending in `\textit{-e}'. These are marked with a superscript `\textit{e}'.
\begin{description}
\item Nearly all nouns ending in `\textit{-ta}' have bases ending in `\textit{-te}', so the superscript `\textit{-e}' is not considered necessary. The few exceptions where the citation and base forms both end in `\textit{-ta}' are marked.
\end{description}
\item Nouns that end in `\textit{-n}' have bases that end in `\textit{-te}' instead.
\begin{description}
\item This suggests that any base form ending in `\textit{-te}' may have a citation form ending in `\textit{n}' or `\textit{ta}'. If there is a consonant before `\textit{-te}' in the base however, the citation form can only end in `\textit{-ta}'.
\end{description}
\item Nouns that end in `\textit{-q}', `\textit{-k}', `\textit{-qw}', or `\textit{-kw}' have bases that end in `\textit{-gh}', `\textit{-g}', `\textit{-ghw}', and `\textit{-w}' respectively.
\item \textsc{exception}: The word `\textit{yuuk}' has the base `\textit{yug}'.
\end{enumerate}

\subsection{Unpossessed Absolutive Plurals and Duals}

\textsc{plurals}: \textbf{${\sim}_\text{sf}\text{-}_\text{w}$:(e)t}

\noindent \textsc{duals}: \textbf{${\sim}_\text{sf}\text{-}_\text{w}$:(e)k}

\begin{enumerate}
\item \textbf{${\sim}_\text{sf}$}: Drop \textit{semi-final} \textbf{e} from the base if it exists (This is an `e' that is followed by another letter).
\begin{description}
\item If dropping semi-final `e' results in a cluster of three consonants within the word or two consonants at the beginning of the word, `e' is \textsc{not} dropped.
\item If the semi-final `e' is dropped and the base has only one full vowel preceding the `e', that full vowel is lengthened and this is called \textbf{e} \textit{hopping}.
\end{description}
\item \textbf{$\text{-}_\text{w}$}: \textit{Weak} consonants are dropped from the base. These \underline{exclude} `\textit{g}', `\textit{ghw}' and `\textit{w}', but `\textit{gh}' may be either weak or strong. A `\textit{gh}' that is preceded by `\textit{e}' or two full vowels is invariably strong, while a `\textit{gh}' that is preceded by one full vowel is most likely weak. Unexpectedly strong `\textit{gh}'s are marked with an * in vocabulary lists.
\item \textbf{(e)t} \textit{or} \textbf{(e)k}: Suffix this `\textit{e}' if the base at this point ends in a strong consonant. Whether this `\textit{e}' is added, always suffix `\textit{t}' or `\textit{k}' at the end.
\item \textbf{:}: \textit{Uvular dropping} is indicated by the colon `:'. If the uvular `\textit{gh}' appears between two short vowels after the above operations, the first of which is a full vowel, the `\textit{gh}' drops, and the `\textit{e}' changes to match the full vowel it is now next to.
\end{enumerate}

\end{document}
